\documentclass{article}

% Packages
\usepackage[utf8]{inputenc}  % Encoding
\usepackage[T1]{fontenc}     % Font encoding
\usepackage{lmodern}         % Better fonts
\usepackage{amsmath, amssymb} % Math symbols
\usepackage{graphicx}        % Images
\usepackage{xcolor}          % Colors
\usepackage{hyperref}        % Hyperlinks
\usepackage{geometry}        % Page layout
\geometry{margin=1in}       % 1-inch margins

\title{A tricky inequality}
\author{Ramansh Sharma \url{rsmath.github.io}}
\date{\today}

\begin{document}

\maketitle

\section{Problem}
Let $a+b+c = 1$. Prove the inequality
\begin{align}
\sqrt{4a + 1} + \sqrt{4b + 1} + \sqrt{4c + 1} \le \sqrt{21}.
\end{align}
%
\section{Proof}
I assume $a, b, c \in \mathbb{R}_{\ge 0}$. The assumption to stay on the positive real is justified below. We perform some preliminary algebra,
\begin{align}
\sqrt{4a + 1} + \sqrt{4b + 1} + \sqrt{4c + 1} \le \sqrt{21}, \\ \nonumber
4a + 4b + 4c + 3 + 2 (\sqrt{(4a+1)(4b+1)} + \sqrt{(4a+1)(4c+1)} + \sqrt{(4c+1)(4b+1)}) \le 21, \\ \nonumber
4 \underbrace{(a + b + c)}_{1} + 3 + 2 (\sqrt{(4a+1)(4b+1)} + \sqrt{(4a+1)(4c+1)} + \sqrt{(4c+1)(4b+1)}) \le 21, \\ \nonumber
\sqrt{(4a+1)(4b+1)} + \sqrt{(4a+1)(4c+1)} + \sqrt{(4c+1)(4b+1)} \le 9.
\end{align}
Simplifying the left side we get
\begin{align}
2 (\sqrt{ab} + \sqrt{bc} + \sqrt{ac}) + 2 (\sqrt{a} + \sqrt{b} + \sqrt{c}) \le 3. \nonumber
\end{align}
Using the general formula $(a+b+c)^2 = a^2+b^2+c^2 + 2(ab+bc+ac)$, we modify the left side and simplify
\begin{align}
(\sqrt{a} + \sqrt{b} + \sqrt{c})^2 \underbrace{- \sqrt{a}^2 - \sqrt{b}^2 - \sqrt{c}^2}_{-1} + 2 (\sqrt{a} + \sqrt{b} + \sqrt{c}) \le 3, \\ \nonumber
(\sqrt{a} + \sqrt{b} + \sqrt{c})^2 + 2 (\sqrt{a} + \sqrt{b} + \sqrt{c}) \le 4. \nonumber
\end{align}
At this point, we use Jensen's inequality for concave functions that says $f(ax + by + cz) \ge af(x) + bf(y) + cf(z)$ as long as $a+b+c=1$. Applied to the square root function here, $\sqrt{a} + \sqrt{b} + \sqrt{c} \le \sqrt{a+b+c}$. So,
\begin{align}
(\sqrt{a+b+c})^2 + 2 (\sqrt{a+b+c}) \le 4, \\ \nonumber
(1)^2 + 2 (1) \le 4, \\ \nonumber
3 \le 4.
\end{align}
Hence, proven.
%
\end{document}
